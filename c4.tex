\documentclass{mwart} % Polska wersja klasy article

\usepackage{polski} % Pozwala na użycie polskiego. Ustawia między innymi fontenc na T1
\usepackage[utf8]{inputenc} % Informuje o kodowaniu
\usepackage{textcomp} % Znaki specjalne takie jak ~
\usepackage{xcolor} % Definicje kolorów

\renewcommand{\labelitemi}{\textbullet} % Zmiana symbolu wliczeń

\usepackage{graphicx}
\graphicspath{ {./Obrazy/} }
% \usepackage{subcaption} % Subfigury
\usepackage{float} % Pozycjonowanie figur
\usepackage{mwe} % Tymczasowe grafiki

\usepackage{listings} % Listingi kodu
\lstset{basicstyle=\ttfamily,
  showstringspaces=false,
  commentstyle=\color{gray},
  keywordstyle=\color{blue}
}

\title{Laboratorium sieci komputerowych - c4 \\ Sieci bezprzewodowe}
\author{Krzysztof Dąbrowski gr. 3}
\date{\today}

\begin{document}
\maketitle{}
\tableofcontents{}
%\newpage

\section{Cel zajęć}
Celem laboratorium jest zbadanie lokalnych sieci radiowych oraz podłączenie i konfiguracja interfejsów radiowych na maszynach z systemami Ubuntu i FreeBSD.

\section{Analiza przestrzeni radiowej}
Przy pomocy aplikacji \textit{Wifi Analyzer} przeskanowałem dostępne sieci radiowe oraz pokrycie poszczególnych kanałów.
Wyniki analizy sieci pokazuje rysunek \ref{fig:wifiAnalizer}.

%TODO: Skompilować gdzieś, gdzie subfigure działa
% \begin{figure}[H]
%     \centering
%     \begin{subfigure}[b]{0.49\textwidth}
%         \centering
%         \includegraphics[width=0.4\textwidth]{Sieci-2,4GH}
%         \caption{Sieci na częstotliwości 2,4GHz}
%     \end{subfigure}
%     \begin{subfigure}[b]{0.49\textwidth}
%         \centering
%         \includegraphics[width=0.4\textwidth]{Sieci-5GH}
%         \caption{Sieci na częstotliwości 5GHz}
%     \end{subfigure}
%     \label{fig:wifiAnalizer}
% \end{figure}

Dodatkowo przeskanowałem dostępne sieci przy pomocy polecenia \texttt{nmcli device wifi list}.

\begin{verbatim}
*  SSID                  MODE   CHAN  RATE    SIGNAL  SECURITY
konferencja           Infra  11    54 Mbit/s  74      WEP
pwwifi-students       Infra  11    54 Mbit/s  30      --
pwwifi2               Infra  11    54 Mbit/s  30      WPA2 802.1X
pwwifi-students       Infra  11    54 Mbit/s  35      --
vlab_net              Infra  11    54 Mbit/s  35      WPA2
konferencja           Infra  11    54 Mbit/s  30      WEP
pwwifi                Infra  11    54 Mbit/s  49      --
ZETIS                 Infra  1     54 Mbit/s  99      WPA2 802.1X
pwwifi-students       Infra  6     54 Mbit/s  34      --
TROL                  Infra  1     54 Mbit/s  29      --
pwwifi2               Infra  1     54 Mbit/s  52      WPA2 802.1X
pwwifi                Infra  11    54 Mbit/s  30      --
pwwifi2               Infra  11    54 Mbit/s  75      WPA2 802.1X
pwwifi2               Infra  6     54 Mbit/s  35      WPA2 802.1X
Sieć Wi-Fi (WE-Lech)  Infra  6     54 Mbit/s  30      WPA2
pwwifi-students       Infra  6     54 Mbit/s  37      --
Stery3                Infra  11    54 Mbit/s  30      WPA1 WPA2
asdf                  Infra  9     54 Mbit/s  49      WPA1 WPA2
pwwifi2               Infra  6     54 Mbit/s  30      WPA2 802.1X
linksys               Infra  3     54 Mbit/s  24      WPA2
konferencja           Infra  1     54 Mbit/s  54      WEP
konferencja           Infra  6     54 Mbit/s  40      WEP
konferencja           Infra  1     54 Mbit/s  37      WEP
konferencja           Infra  6     54 Mbit/s  34      WEP
is_wifi               Infra  4     54 Mbit/s  30      WEP
konferencja           Infra  6     54 Mbit/s  30      WEP
pwwifi                Infra  1     54 Mbit/s  54      --
pwwifi-students       Infra  1     54 Mbit/s  49      --
pwwifi                Infra  6     54 Mbit/s  44      --
pwwifi                Infra  1     54 Mbit/s  37      --
pwwifi-students       Infra  11    54 Mbit/s  30      --
pwwifi2               Infra  1     54 Mbit/s  42      WPA2 802.1X
pwwifi2               Infra  6     54 Mbit/s  32      WPA2 802.1X
konferencja           Infra  1     54 Mbit/s  20      WEP
pwwifi                Infra  1     54 Mbit/s  37      --
pwwifi-students       Infra  1     54 Mbit/s  24      --
pwwifi-students       Infra  1     54 Mbit/s  20      --
\end{verbatim}

\section{Schemat sieci}
Strukturę urządzeń w sieci przedstawia rysunek \ref{fig:SchematSieci}.

\begin{figure}[H]
  \centering
  \includegraphics[width=\textwidth]{SchematSieci}
  
  \caption{Schemat sieci}
  \label{fig:SchematSieci}
\end{figure}

\section{Podłączenie do sieci wifi w środowisku graficznym}
W celu przyłączenia do sieci skorzystam z nakładki graficznej na program \textit{NetworkManager} wbudowanej w system Ubuntu.
\vspace{5 mm}

Przed podłączeniem sprawdziłem stan interfejsu radiowego poleceniem \texttt{ip~a}.
\begin{verbatim}
    ip a

    4: wlp2s0: <NO-CARRIER,BROADCAST,MULTICAST,UP> mtu 1500 qdisc mq state DOWN group default qlen 1000
    link/ether 00:24:d7:92:0e:dc brd ff:ff:ff:ff:ff:ff
\end{verbatim}

Oraz tablicę tras, poleceniem \texttt{ip r}.
\begin{verbatim}
    ip r 

    default via 10.146.146.5 dev eno2
    10.146.0.0/16 dev eno2  proto kernel  scope link  src 10.146.225.1
\end{verbatim}

Z otrzymanych wyników wiać, że interfejs radiowy jest \textbf{nieaktywny} a trasa domyślna wiedzie przez interfejs fizyczny.
\vspace{5 mm}

Po podłączeniu do sieci \textbf{ZETIS} wyniki tych poleceń wyglądały następująco:
\begin{verbatim}
    ip a

    4: wlp2s0: <BROADCAST,MULTICAST,UP,LOWER_UP> mtu 1500 qdisc mq state UP group default qlen 1000
    link/ether 00:24:d7:7d:ba:8c brd ff:ff:ff:ff:ff:ff
    inet 10.68.17.233/16 brd 10.68.255.255 scope global dynamic wlp2s0
       valid_lft 3601sec preferred_lft 3601sec
    inet6 fe80::224:d7ff:fe7d:ba8c/64 scope link 
       valid_lft forever preferred_lft forever
\end{verbatim}

\begin{verbatim}
    ip r 

    default via 10.146.146.5 dev eno2 
    default via 10.68.0.1 dev wlp2s0  proto static  metric 600 
    10.68.0.0/16 dev wlp2s0  proto kernel  scope link  src 10.68.17.233  metric 600 
    10.146.0.0/16 dev eno2  proto kernel  scope link  src 10.146.225.3 
    169.254.0.0/16 dev wlp2s0  scope link  metric 1000 
    192.0.2.4 via 10.68.0.1 dev wlp2s0  proto dhcp  metric 600 
\end{verbatim}

Widać, że interfejs radiowy \texttt{wlp2s0} jest teraz włączony oraz skonfigurowany.
Do tablicy tras została dodana nowa domyślna trasa prowadząca przez interfejs radiowy.
\vspace{5 mm}

Dodatkowo pobrałem logi z serwera RADIUS połączeniem komend \texttt{ssh ldap grep -w \$USER /var/log/radiusd | tail -2}.
\begin{verbatim}
    ssh ldap grep -w \$USER /var/log/radiusd | tail -2

    Mon May 13 17:01:43 2019 : Auth: (2156)   Login OK: [dabrowk1] 
        (from client ap225 port 0 via TLS tunnel)
    Mon May 13 17:01:43 2019 : Auth: (2156) Login OK: [dabrowk1]
        (from client ap225 port 0 cli 00-22-3F-01-F9-12)
\end{verbatim}
Z zebranych logów wynika, że serwer RADIUS zaakceptował podane dane dostępowe.

\end{document}